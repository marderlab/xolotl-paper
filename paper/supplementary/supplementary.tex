%%%%%%%%%%%%%%%%%%%%%%%%%%%%%%%%%%%%%%%%%%%%%%%%%%%%%%%%%%%%%%%%%%%%%%%%%%%%%%%%%%%%%%%%%%%%%%%%%%%%%%%%%%%%%%%%%%%%%%%%%%%%%%%%%%%%%%%%%%%%%%%%%%%%%%%%%%%
% This is just an example/guide for you to refer to when producing your supplementary material for your Frontiers article.                                 %
%%%%%%%%%%%%%%%%%%%%%%%%%%%%%%%%%%%%%%%%%%%%%%%%%%%%%%%%%%%%%%%%%%%%%%%%%%%%%%%%%%%%%%%%%%%%%%%%%%%%%%%%%%%%%%%%%%%%%%%%%%%%%%%%%%%%%%%%%%%%%%%%%%%%%%%%%%%

%%% Version 2.5 Generated 2018/06/15 %%%
%%% You will need to have the following packages installed: datetime, fmtcount, etoolbox, fcprefix, which are normally inlcuded in WinEdt. %%%
%%% In http://www.ctan.org/ you can find the packages and how to install them, if necessary. %%%
%%%  NB logo1.jpg is required in the path in order to correctly compile front page header %%%

\documentclass[utf8]{frontiers_suppmat} % for all articles
\usepackage{url,hyperref,lineno,microtype}
\usepackage[onehalfspacing]{setspace}

% bibliography
\usepackage[english]{babel}
\usepackage{csquotes}
\usepackage[
backend=biber, 
bibencoding=utf8, 
style=authoryear, 
date=year, 
maxbibnames=6, 
minbibnames=6, 
url=false, 
isbn=false, 
eprint=false,
uniquename = false, 
autocite=inline]{biblatex}
\addbibresource{conductances.bib}

% make pretty MATLAB code
\usepackage{listings}
\usepackage{matlab-prettifier}
\lstMakeShortInline[style=Matlab-editor]"



% Leave a blank line between paragraphs instead of using \\

\begin{document}
\onecolumn
\firstpage{1}

\title[Supplementary Material]{{\helveticaitalic{Supplementary Material}}:
\\ \helvetica{Xolotl: An Intuitive and Approachable Neuron \& Network Simulator in MATLAB}} %Please insert the title of your article here


\maketitle


\section{Database of Conductances}

	Dynamics for model conductances in the following papers have been implemented in "xolotl".
	
	\begin{enumerate}
		\item EAGes, EAGmut, EAGwt (\cite{bronkRegulationEagCa22018})
		\item A-type (\cite{brookingsAutomaticParameterEstimation2014})
		\item Proctolin modulatory input current (\cite{caplanManyParameterSets2014})
		\item CaT, H, Kd, NaV (\cite{dethierPositiveFeedbackCellular2015})
		\item CaL, CaPump, KCa, Kd, NaV (\cite{drionDopaminePacemakerNeuron2017})
		\item Ap, At, Cal, KCa, KCaT, Proc (\cite{goldmanGlobalStructureRobustness2001})
		\item Shaker (\cite{hardieNovelPotassiumChannels1991})
		\item Novel, Shab, Shaker (\cite{herasModulationVoltagedependentConductances2018})
		\item A-type, CaS, CaT, H, KCa, Kd, NaV (\cite{kisperskyIncreaseSodiumConductance2012})
		\item NaP, NaT (\cite{linActivityDependentAlternativeSplicing2012})
		\item A-type, CaS, CaT, H, KCa, Kd, NaV (\cite{liuModelNeuronActivityDependent1998})
		\item \cite{liuModelNeuronActivityDependent1998} conductances with cached gating functions
		\item \cite{liuModelNeuronActivityDependent1998} conductances with forward Euler integration
		\item \cite{liuModelNeuronActivityDependent1998} conductances with temperature dependence
		\item Kd and NaV conductances from Int1, LG, and MCN1 cells (\cite{nadimFrequencyRegulationSlow1998})
		\item Calcium current (\cite{nadimSynapticDepressionCreates1999})
		\item Drosophila NaV (\cite{odowdVoltageclampAnalysisSodium1988})
		\item A-type, CaS, CaT, H, KCa, Kd, NaV (\cite{prinzAlternativeHandtuningConductancebased2003})
		\item \cite{prinzAlternativeHandtuningConductancebased2003} conductances with cached gating functions
		\item \cite{prinzAlternativeHandtuningConductancebased2003} conductances with temperature dependence
		\item Kd and NaV from Int1 cells and modulatory input and transient LTS conductances from LG cells (\cite{rodriguezConvergentRhythmGeneration2013})
		\item Proctolin modulatory input (\cite{sharpDynamicClampComputergenerated1993})
		\item Cal and Kd from AB-PD, LP, and PY cells (\cite{soto-trevinoActivitydependentModificationInhibitory2001})
		\item A-type, CaT, and KCa from AB and PD cells, generic H, Kd, NaP, NaV, and proctolin conductances (\cite{soto-trevinoComputationalModelElectrically2005})
		\item Proctolin modulatory input (\cite{swensenModulatorsConvergentCellular2001})
		\item A-type, CaS, CaT, H, KCa, Kd, NaP, NaV (\cite{turrigianoSelectiveRegulationCurrent1995})
		\item Drosophila NaV (\cite{wicherNonsynapticIonChannels2001})
	\end{enumerate}

\section{Parameters for Simulations}

	Two single-compartment models were simulated in this paper. The first, a Hodgkin-Huxley model with three conductances ("NaV", "Kd", and "Leak") and injected current was simulated for Figure 1 and 7. The second is a stomatogastric neuron model with eight conductances ("NaV", "CaT", "CaS", "ACurrent", "KCa", "Kd", "HCurrent", "Leak") and was simulated for Figure 3 and 7. Both models have dynamics as described in \cite{liuModelNeuronActivityDependent1998}.
	
	\subsection{Parameters for Hodgkin-Huxley model} 
	
		\begin{center}
			\begin{tabular}{|l|r|l|}
				\hline 
				\textbf{Parameter Name} & \textbf{Value} & \textbf{Units} \\ 
				\hline 
				Membrane capacitance ("HH.Cm") & 10 & ${nF}/{mm^2}$ \\ 
				\hline 
				Surface area ("HH.A") & 0.01 & $mm^2$ \\ 
				\hline 
				Fast sodium maximal conductance ("HH.NaV.gbar") & 1000 & $\mu S/mm^2$ \\ 
				\hline 
				Delayed rectifier maximal conductance ("HH.Kd.gbar") & 300 & $\mu S/mm^2$ \\ 
				\hline 
				Passive leak maximal conductance ("HH.Leak.gbar") & 1 & $\mu S/mm^2$ \\ 
				\hline 
				Sodium reversal potential ("HH.NaV.E") & 50 & $mV$ \\ 
				\hline 
				Potassium reversal potential ("HH.Kd.E") & -80 & $mV$ \\ 
				\hline
				Leak reversal potential ("HH.Leak.E") & -50 & $mV$ \\
				\hline
				Injected current ("I_ext") & 0.2 & $nA$ \\
				\hline 
			\end{tabular} 
		\end{center}
	
	\subsection{Parameters for Stomatogastric Model}
	
		\begin{center}
			\begin{tabular}{|l|r|l|}
				\hline 
				\textbf{Parameter Name} & \textbf{Value} & \textbf{Units} \\ 
				\hline 
				Membrane capacitance ("AB.Cm") & 10 & ${nF}/{mm^2}$ \\ 
				\hline 
				Surface area ("AB.A") & 0.0628 & $mm^2$ \\
				\hline
				Calcium buffering $\phi$ ("AB.phi") & 90 & $\mu M / nA$ \\
				\hline
				Calcium buffering shell volume ("AB.vol") & 0.0628 & $mm^3$ \\
				\hline 
				Fast sodium maximal conductance ("AB.NaV.gbar") & 00 & $\mu S/mm^2$ \\ 
				\hline
				Transient calcium maximal conductance ("AB.CaT.gbar") & 00 & $\mu S/mm^2$ \\ 
				\hline 
				Slow calcium maximal conductance ("AB.CaS.gbar") & 00 & $\mu S/mm^2$ \\ 
				\hline 
				Transient potassium maximal conductance ("AB.ACurrent.gbar") & 00 & $\mu S/mm^2$ \\ 
				\hline 
				Calcium-gated potassium maximal conductance ("AB.KCa.gbar") & 00 & $\mu S/mm^2$ \\ 
				\hline 
				Delayed rectifier maximal conductance ("AB.Kd.gbar") & 00 & $\mu S/mm^2$ \\ 
				\hline
				Hyperpolarization-activated maximal conductance ("AB.HCurrent.gbar") & 00 & $\mu S/mm^2$ \\ 
				\hline 
				Passive leak maximal conductance ("AB.Leak.gbar") & 00 & $\mu S/mm^2$ \\ 
				\hline 
				Sodium reversal potential ("AB.NaV.E") & 30 & $mV$ \\ 
				\hline 
				Potassium reversal potential ("AB.Kd.E") & -80 & $mV$ \\ 
				\hline
				Hyperpolarization-activated reversal potential ("AB.HCurrent.E") & -20 & $mV$ \\
				\hline
				Leak reversal potential ("AB.Leak.E") & -50 & $mV$ \\
				\hline
			\end{tabular} 
		\end{center}
	
	\subsection{Parameters for Network Model}
		The network model displayed in Figure 4 is described in \cite{prinzAlternativeHandtuningConductancebased2003, prinzSimilarNetworkActivity2004}. It is comprised of three compartments "AB", "LP", and "PY" with the same dynamics but differing parameters and synaptic inputs.
		
		\begin{center}
			\begin{tabular}{|l|c|c|c|c|}
				\hline 
				\textbf{Parameter Name} & \textbf{AB} & \textbf{LP} & \textbf{PY} & \textbf{Units} \\ 
				\hline 
				Fast sodium maximal conductance & 1000 & 1000 & 1000 & $\mu S/mm^2$ \\ 
				\hline 
				Transient calcium maximal conductance & 25 & 0 & 24 & $\mu S/mm^2$ \\  
				\hline 
				Slow calcium maximal conductance & 60 & 40 & 20 & $\mu S/mm^2$ \\  
				\hline 
				Transient calcium maximal conductance & 500 & 200 & 500 & $\mu S/mm^2$ \\  
				\hline 
				Calcium-gated potassium maximal conductance & 50 & 0 & 0 & $\mu S/mm^2$ \\  
				\hline 
				Delayed rectifier maximal conductance & 1000 & 250 & 1250 & $\mu S/mm^2$ \\  
				\hline 
				Hyperpolarization-activated maximal conductance & 0.1 & 0.5 & 0.5 & $\mu S/mm^2$ \\  
				\hline 
				Leak maximal conductance & 0 & 0.3 & 0.1 & $\mu S/mm^2$ \\  
				\hline 
				Sodium reversal potential & 50 & 50 & 50 & $mV$ \\ 
				\hline 
				Potassium reversal potential & -80 & -80 & -80 & $mV$ \\ 
				\hline
				Hyperpolarization-activated reversal potential & -20 & -20 & -20 & $mV$ \\
				\hline
				Leak reversal potential & -50 & -50 & -50 & $mV$ \\
				\hline
			\end{tabular}
		\end{center}
	
		There are seven synapses in the model of two types: glutamatergic ("Glut") and cholinergic ("Chol") with differing dynamics. The synapse produces a current in the postsynaptic compartment based on the membrane potential in the presynaptic compartment. 
		
		\begin{center}
			\begin{tabular}{|c|c|c|c|}
				\hline
				\textbf{Presynaptic} & \textbf{Postsynaptic} & \textbf{Type} & \textbf{Maximal Conductance} ($\mu S/mm^2$) \\
				\hline
				AB & LP & Chol & 30 \\ \hline
				AB & PY & Chol & 3 \\ \hline
				AB & LP & Glut & 30 \\ \hline
				AB & PY & Glut & 10 \\ \hline
				LP & PY & Glut & 1 \\ \hline
				PY & LP & Glut & 30 \\ \hline
				LP & AB & Glut & 30 \\ \hline
			\end{tabular}
		\end{center} 


\printbibliography

\end{document}
